\documentclass{article}

% Language setting
\usepackage[polish]{babel}

% Set page size and margins
\usepackage[a4paper,top=2cm,bottom=2cm,left=3cm,right=3cm,marginparwidth=1.75cm]{geometry}
\usepackage[T1]{fontenc}

% Useful packages
\usepackage{amsmath}
\usepackage{graphicx}
\usepackage{subcaption}
\usepackage[colorlinks=true, allcolors=blue]{hyperref}
\usepackage{float}
\usepackage{listings}

\title{CAD/CAE - zadanie 5}
\author{Iwo Szczepaniak}

\begin{document}
\maketitle

\section{Zmiany w kodzie}
W ramach zadania zmodyfikowano kod źródłowy w celu obsługi obrazu wejściowego oraz przygotowania wektorów węzłowych dla algorytmu B-spline. Poniżej przedstawiono kluczowe fragmenty implementacji:
\begin{verbatim}
img = imread("gear-icon.png");

[rows, cols, channels] = size(img);
padded_img = zeros(rows + 2, cols + 2, channels);

for c = 1:channels
  padded_img(2:end-1, 2:end-1, c) = img(:, :, c);
end

img = padded_img;

isize = size(img, 1);
knot_vectorx=0:isize-1;
knot_vectorx(1:3) = 0;

for i = 4:isize
  knot_vectorx(i) = i-3;
end

knot_vectorx(end-2 : end) = isize-5;
knot_vectory = knot_vectorx;

%knot = simple_knot(16, 2);

% Input data
knot = knot_vectorx;     
% knot vector (5,2)
\end{verbatim}

\section{Pomocnicza funkcja make\_gif.m}
Dla wizualizacji wyników, stworzono funkcję pomocniczą make\_gif.m, która łączy sekwencję wygenerowanych obrazów w animację GIF. Funkcja przetwarza obrazy od t=0 do t=100:
\begin{verbatim}
outputGif = 'output.gif';

for i = 0:1:100
    filename = sprintf('out_%d.png', i);
    img = imread(filename);
    [A, map] = rgb2ind(img, 256);
    
    if i == 0
        imwrite(A, map, outputGif, 'GIF', 'LoopCount', inf, 'DelayTime', 0.1);
    else
        imwrite(A, map, outputGif, 'GIF', 'WriteMode', 'append', 'DelayTime', 0.1);
    end
end
\end{verbatim}

\section{Obraz Wejściowy}
Jako dane wejściowe wykorzystano czarno-białą bitmapę przedstawiającą zębatkę. Obraz ten posłużył jako warunek początkowy dla symulacji:

\begin{figure}[H]
    \centering
    \includegraphics[width=0.8\linewidth]{gear-icon.png}
    \caption{Bitmapa zębatki}
    \label{fig:height-map}
\end{figure}

\section{Wyniki symulacji}
Przeprowadzona symulacja pokazuje ewolucję kształtu zębatki w czasie. Poniżej przedstawiono wyniki w postaci sekwencji obrazów oraz animacji:

\subsection{Obrazy w różnych krokach czasowych}
Wybrano sześć reprezentatywnych momentów czasowych (t = 0, 20, 40, 60, 80, 100), aby zobrazować postęp transformacji:

\begin{figure}[H]
    \centering
    \begin{subfigure}[b]{0.3\textwidth}
        \includegraphics[width=\textwidth]{out_0.png}
        \caption{t = 0}
    \end{subfigure}
    \begin{subfigure}[b]{0.3\textwidth}
        \includegraphics[width=\textwidth]{out_20.png}
        \caption{t = 20}
    \end{subfigure}
    \begin{subfigure}[b]{0.3\textwidth}
        \includegraphics[width=\textwidth]{out_40.png}
        \caption{t = 40}
    \end{subfigure}
    \begin{subfigure}[b]{0.3\textwidth}
        \includegraphics[width=\textwidth]{out_60.png}
        \caption{t = 60}
    \end{subfigure}
    \begin{subfigure}[b]{0.3\textwidth}
        \includegraphics[width=\textwidth]{out_80.png}
        \caption{t = 80}
    \end{subfigure}
    \begin{subfigure}[b]{0.3\textwidth}
        \includegraphics[width=\textwidth]{out_100.png}
        \caption{t = 100}
    \end{subfigure}
    \caption{Ewolucja animacji w zależności od kroku algorytmu}
    \label{fig:temperature-evolution}
\end{figure}

\subsection{Animacja}
Dla pełniejszego zobrazowania procesu, wszystkie kroki czasowe połączono w płynną animację. Pełna animacja procesu została zapisana w pliku output.gif, który pokazuje ewolucję rozkładu w czasie ciągłym.

\end{document}